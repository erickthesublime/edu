\setcounter{chapter}{1}
\centerline{Quick Description}
 Reading literature prior to writing a paper is essential, thus this could very well be a starting place for our group. 
 This document describes the direction it is logical to go in from starting at the \emph{literature review} section. 
 I prefer traditional research publication anatomy, starting with an abstract with the goal of the paper right away. 
 If there is some disagreement about how to label sections, I think attempting to directly quote the grading criteria would be the most beneficial. 
 \\

\centerline{If any text is marked in \textbf{bold} along with a \textbf{(\# of points)} it is because it is directly from the grading criteria.
}

\centerline{Or if any text has red exponents \textcolor{red}{A, B, C, D, E, X, Y, Z, P, J, O, B, S} it also relates to the evaluation criteria.}

\centerline{(I updated the grading criteria on Github to match.)}

\section{Research Anatomy}
\subsection{Introduction \textcolor{BurntOrange}{2$^{\text{nd}}$ Chronologically}}

Should have our \emph{literature review} in it, make note of key ideas and always keep urls for sources. 
  \begin{noot}{The first characters of our paper!}
  *Key ideas from reading literature*

  References:

	  1) www.urls-for-the-literature-above.com
\end{noot}
That will get us started, and from there we can build up the problem question before proving our solution.
\begin{enumerate}
	\item What has led up to the question being asked?
	\begin{enumerate}
		\item It'd be a good idea to begin just before the scientific method started to be used to answer this question
		\begin{enumerate}
			\item Ever since electricity became mainstream$\to$ xyz scientific methods have been used to measure it. 
		\end{enumerate}
	\end{enumerate}
	\item It is likely that any problem they could possibly ask us to answer, many people have already provided many solutions. Include limitations of them. 
		\begin{enumerate}
			\item Discuss evaluation criteria set forth by other authors who are interested in this question.
				\begin{enumerate}
					\item Show these criteria support the process of studying and answering the question. \hfill\textcolor{red}{C}
				\end{enumerate}
		\end{enumerate}
		\begin{enumerate}
			\item Briefly give an overview of all the different solutions to the question
\begin{enumerate}
	\item renewable energy? solar,wind,hydro,biomass $\to$ solar's is the best *boom* winning entry.
\end{enumerate}
		\end{enumerate}
	\item The careers of those people who have previously provided solutions. \hfill \textcolor{red}{J}
\begin{enumerate}
	\item Include salary and job environment \hfill \textcolor{red}{O}
	\item Exploration of careers including academic preparation\hfill \textcolor{red}{B}
	\item Extend this well beyond expectations! \hfill\textcolor{red}{S}
\end{enumerate}
	\item We need two people to write career biography's about, and who used mathematical models to come up with a answer to the same question we're answering. \hfill\textcolor{red}{S}
	\item Math time!
		\begin{enumerate}
			\item Quantify the parameters that we would need to consider in order to mathematically deduce some conclusion from data
		\begin{enumerate}
			\item *this* thing is central to the question, so is *this other* thing. \hfill \textcolor{red}{X}
\item *XYZ* government bureau records *this* data
		\end{enumerate}
	\item Discuss what different operations could possibly be done with these parameters/discuss possible data sets. \hfill\textcolor{red}{B}
		\begin{enumerate}
			\item Well, *this* thing $\times $ *this other* thing allows us to deduce *a third* thing
			\item Or, \hfill\textcolor{red}{Y}\[
			\frac{\text{*this other* thing}}{\text{*a third* thing}}= \text{*this* thing}
				.\] 
		\item Formulate one or more mathematical models that accurately answer the question \hfill\textcolor{red}{Z}
	\item Describe how to extend this to get near cutting edge research \hfill \textcolor{red}{P}
		\end{enumerate}
			\item I think it would be really benefical to look through textbooks such as calculus and chemistry for real-world application problems. 
				Often times these problems have mathematical models that describe real world processes, and often times they can be advanced.
		\end{enumerate}
	\item Now we can state our thesis! \hfill\textcolor{red}{A}
		\begin{enumerate}
			\item This will be a detailed description of our findings, with a focus on the data that we 1. used and 2. generated through the mathematical model.
		\end{enumerate}
	\end{enumerate}
\subsection{Abstract \textcolor{BurntOrange}{1$^{\text{st}}$ Chronologically}}
By deducing our thesis from mathematical models, now we can go back and create the abstract. Traditionally it would contain what we studied, the outcome of our study, and what comes next now this information exists.
\subsection{Discussion}
\begin{enumerate}
	\item Additional related ideas setting the focus of this work.  \hfill\textcolor{red}{E}
	\item Additional ways our mathematical models could potentially serve deeper uses. (Maybe being repackaged to solve other problems)
\end{enumerate}

\subsection{References \textcolor{BurntOrange}{Last}}
\begin{enumerate}
	\item Reliable sources \hfill\textcolor{red}{D} 
\end{enumerate}
